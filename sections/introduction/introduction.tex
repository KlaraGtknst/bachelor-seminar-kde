\section{Introduction}

In order to find methods of computation requiring little power consumption, 
researchers have started creating structures modelled on the neurons of the brain.
The authors of \cite{SNN} found that \acp{SNN} were suitable means to tackle this task.
\acp{SNN} differ from normal \acp{ANN} in terms of their architecture and learning method.
Their inputs are 1-bit spike trains as opposed to 32- or 64-bit messages of \acp{ANN}.
Moreover, instead of backpropagation used for \acp{ANN}, many \ac{SNN} models have different learning rules to optimise their weight, 
such as \acfi{STDP} with exponential time dependence.
The authors of \cite{SNN} used \acfi{LIF} neurons and lateral inhibition.
Hence, the approach models the leak of current of real neurons, as well as competition among the neurons.

Applications for this approach include pattern recognition presented in \cite{SNN} and object shape recognition from \cite{object_detection_SNN}.
In both cases the accuracy is surprisingly high for an unsupervised method.

This paper is structured as followed:
In \autoref{sec:main_part} the tackled problem regarding \acp{SNN}, context-specific terms, the network architecture,
 as well as the approach itself are described.

\textcolor{red}{
In \autoref{sec:result} the results of tests approach is presented.
Certain methods used in the approach of \cite{SNN} are evaluated in \autoref{sec:evaluation}.
Different methods to train \acp{SNN} are described and compared in \autoref{sec:comparison}.
The paper concludes with a outlook in \autoref{sec:conclusion}.}