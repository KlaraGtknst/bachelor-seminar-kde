\section{Introduction}

In order to find methods of computation requiring little power consumption, 
researchers have started creating structures modelled on the neurons of the brain.
\acp{SNN} are allegedly suitable means to tackle this task \cite{SNN}.
The model presented in the following uses \acfi{LIF} neurons and lateral inhibition \cite{SNN}.
Hence, the approach models the leak of current of real neurons, as well as competition among the neurons.

Applications for \acp{SNN} include pattern recognition \cite{SNN} and object shape recognition \cite{object_detection_SNN,multi_scale_STDP}.
In both cases, the accuracy is surprisingly high for an unsupervised method.

This paper is structured as followed:
In \autoref{subsec:problem} the tackled problem regarding \acp{SNN} is outlined.
\autoref{subsec:methods} covers context-specific terms, as well as the approach itself.
The network architecture is described in \autoref{subsec:architecture}.
In \autoref{sec:result} the results of tests regarding performance, optimal parameter choice and other metrics of similar approaches are presented.
Methods to train \acp{SNN}, which differ more than the ones outlined in \autoref{sec:result} are described and compared in \autoref{sec:comparison}.
The paper concludes with an outlook in \autoref{sec:conclusion}.