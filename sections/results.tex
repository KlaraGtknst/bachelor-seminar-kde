\section{Results}
\label{sec:result}
% STDP-like paper
The model was evaluated on the MNIST dataset.
According to the authors of \cite{STDP_like}, the dataset is a suitable Benchmark for the \ac{SNN} model, 
since it provides different difficulty levels of categorization.
The authors of \cite{STDP_like} emphasise that the MNIST dataset less complex than biological vision.

The addition to accuracy with regard to the classification of the MNIST dataset impulses \cite{STDP_like} presents \ac{RT} distributions.
\ac{RT} is defined as the time between the presentation of a stimulus and the response (i.e. the first pool to reach the descision pool).
Usually, the \ac{RT} of misclassified digits is higher than the \ac{RT} of correctly classified digits.
However, the network also makes fast errors.
The results were verified with the Kolmogorov-Smirnov test, 
indicating that the correctly and misclassified \ac{RT} distributions are significantly different, 
whereas as the distributions for stimuli from the training and test set were not.

The authors of \cite{STDP_like} compare the performance of the \ac{SNN} model for different training set sizes $n_{train}$.
They find that the network is able to generalize well if the training set size is sufficiently large.
The optimum training set size $n_{train} = 1000$ produced not remarkably higher misclassifaction rates than bigger training set sizes.
The most frequent misclassification was the digit nine as a zero.


% Original STDP paper
In \cite{SNN} the authors compare the performance of the \ac{SNN} model consisting of different numbers of excitatory neurons and different learning rules.
The visualizations suggest that highest number of excitatory neurons (i.e. 6400) produces the best results for all learning rules.
The authors also study propose possible reasons for the distribution of misclassified digits.
The most common misclassification was the digit four as a nine.

The authors of \cite{SNN} include a table visualizing performances of different \acp{SNN}.
Rate-based learning methods adchieve the best results.


Benchmark\\
zugrundeliegende Daten