\newcommand\rbModel{Rate-based model}
\newcommand\sbModel{Spike-based model}
\subsection{Terms}
\label{subsec:terms}

In this section, technical terms specific to the domain are defined and briefly discussed.

\subsubsection{Spike trains}
are defined as multiple spikes.
\textcolor{red}{
They are encoded as a 1-bit array, where each entry indicates whether a spike (1) or no spike (0) occurred at a certain point in time.
}

\subsubsection{\rbModel{}}
According to \cite{spike_vs_rate} there are two central beliefs with regard to the question of how the neurons communicate with each other.
The firing rate of a neuron is an abstract measurement of the average number of spikes per duration/ neuron/ trial.
The \rbModel{} believes that the firing rate captures most of the information, hence the timing of spikes is meaningless.
Rate-based learning uses backpropagation during training, but converts the \ac{ANN} to a \ac{SNN} afterwards. 
According to \cite{SNN} and \cite{STDP_like} the usage of backpropagation is biologically unrealistic.

\subsubsection{\sbModel{}}
According to the \sbModel{}, the firing rate is not sufficient to describe the neural activity.
The spike timing defines spike trains and individual spikes.
As stated in \cite{spike_vs_rate}, the \rbModel{} assumption is stronger than the one of the \sbModel{}.
