\section{Conclusion and Outlook}
\label{sec:conclusion}

The study of \acp{SNN}, especially with regard to efficient learning procedures poses a interesting insight in the possibilities of modeling
systems on the human brain.
Concerning the difficulties of training a model which works with input values and their temporal dependencies researchers have already made 
impressive efforts.
The work of \cite{SNN} proposes an unsupervised approach with formidable performace in a variety of situations.
The authors also emphasise the energy efficience of \acp{SNN} on neuromorphic hardware.

Besides this example of an unsupervised approach there is a variety of supervised methods to train a \ac{SNN}, 
for instance \ac{ANN}-\ac{SNN} conversion from \cite{DIET_SNN}.
Even though existing approaches performance well, the authors emphasise shortcomings and potential of further research.
\textcolor{red}{Bsp. offene Fragen}

wdhl wie gut es ist\\
was kommt noch?\\
wie zu verbessern?